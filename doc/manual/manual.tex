\documentclass[a4paper]{article}
\usepackage{vmargin}
\usepackage{url}

\newif\ifpdf\ifx\pdfoutput\undefined\pdffalse\else\pdfoutput=1\pdftrue\fi

%% PDF options from Chicken Scheme compiler's manual.
%% Only use hyperref for the PDF output otherwise it puts ugly
%% underlines in the TOC.
\ifpdf
\usepackage[bookmarks=true,bookmarksnumbered=true,bookmarksopen=false,colorlinks=true,pdfpagemode=none,plainpages=false]{hyperref}
\fi


\begin{document}


\newcommand{\key}[1]{\fbox{\texttt{#1}}}


\title{\textbf{Red Pixel 2}}
\date{\textbf{Demo 2}}
\author{Peter Wang\\
\texttt{tjaden@users.sf.net}}
\maketitle

\tableofcontents
\newpage


%----------------------------------------------------------------------
\section{Introduction}

Red Pixel 2 is a platform deathmatch game, playable by two or more
players on a Local Area Network (LAN).

At the beginning of a match players are dropped into a dark 2d map
with only a torch and a blaster (the weakest weapon available).
Players must collect better weapons and ammunition, search out their
opponents, and then shoot them to a bloody pulp.  Dead players will be
respawned, but lose all the possessions they had collected.  Scoring
is based on the number of kills made.


\subsection{System requirements}

\begin{itemize}

\item Processor: i686-ish machine with a clock speed of around 400 MHz
  is probably minimum.  i686 means Pentium Pro or above.

\item Operating system: Windows 9x/ME or Linux.

  It has been reported that Windows XP doesn't work well in
  client-server mode (although it works in client-only mode).  I
  presume the same of Windows NT and Windows 2000.  I don't have
  access to those operating systems, so I don't know when the problem
  will be fixed.

\item Network: Machines must be on a LAN with IP.  High speed Internet
  connections may work, but hasn't been tested.

\item Keyboard, mouse, sound card, etc.

\end{itemize}



%----------------------------------------------------------------------
\section{Entering the game}


\subsection{Starting the server}

A Red Pixel 2 game setup consists of a single server and two or more
clients connected to it.  The server hosts the game and handles most
of the simulation work.  Clients connect to the server to interact
with the simulation.

To start a game, a server must be started first.  In this manual we
will assume the machine running the server will also run a client
(although it is possible to run a dedicated server if you so wish).

In the main menu, select \textsf{Multiplayer} then
\textsf{Client-Server}.  After entering a nickname for yourself select
\textsf{Ok} and you will be placed into the lobby.  At this stage you
will probably want to wait for clients to join your match (although
clients can join at any time).

(If you wish to run two or more different Red Pixel 2 games on the
same LAN, you will need to select a different IP port before pressing
\textsf{Ok}.  I will assume then that you know what you are doing.)


\subsection{Starting a client}

On the client machine, select \textsf{Multiplayer} then
\textsf{Client}.  Enter a nickname for yourself, and type in the name
of the server machine or its IP address.  (Sorry, there is no
broadcast feature.)  You should either be placed into the lobby, or if
you joined in the middle of a game, you will be dropped directly into
the map.


\subsection{The lobby}

In the lobby you can type messages to other clients in the lower left
corner.

The client-server can select the next map to play by clicking on the
small button next to the name of the current map (in the top right
corner).  Once a map has been chosen, the server can start the game by
pressing \textsf{Go!}.

You can click on the small icons which represent each connected
client, and some information about each client will be displayed.  If
it's necessary, the client-server can forcefully disconnect a client
by pressing the \textsf{Kick} button.

Once in the game, the client-server can return to the lobby by
pressing \key{Esc}.  A new map can then be selected, or the game
ended.  If you wish to return to the game without starting a new map,
simply press \key{Esc}.


\subsection{End of match}

At this stage, Red Pixel 2 does not enforce any notions of when a
match is finished.  The server can end the match by returning to the
lobby and pressing the \textsf{Disconnect} button.



%----------------------------------------------------------------------
\section{Playing the game}


\subsection{Controls}

The player moves his/her character around using the \key{W}, \key{A},
\key{S}, \key{D} keys, or using the cursor keys.  Weapons can be
selected directly with the number keys (\key{1} to \key{9}), or you
can cycle through them using the \key{Q} and \key{E}.  If your mouse
has a wheel, scrolling the mouse wheel will also cycle through the
weapons.

Aim the selected weapon using the mouse, and hold down the left mouse
button to fire.  Notice that by pushing the mouse cursor further from
your character you can see further in that direction.

You can drop collected mines by pressing the right mouse button.
Mines are armed soon after dropping.

When you die, press the \key{Space bar} to respawn.

Press \key{Tab} to view everyone's scores.  The total number of kills
(``frags'') is displayed.  A second number (in brackets) may also be
present: this is the number of kills made since this map was started.
Committing suicide will decrease your score.

You can type text messages to other players by pressing \key{Enter}.

Clients can press \key{Esc} during a game to quit.  Client-servers can
press \key{Esc} during a game to return to the lobby.


\subsection{Weapons}

\subsubsection*{Blaster}

The blaster is the weakest of the available weapons.

\noindent
\emph{Ammunition}: none.

\subsubsection*{AK-47}

A rapid firing gun which is slightly better than a blaster.

\noindent
\emph{Ammunition}: bullets.

\subsubsection*{Shotgun}

The shotgun is a powerful weapon that sprays shots in various
directions.  It is particularly good at close range.

\noindent
\emph{Ammunition}: shells.

\subsubsection*{Vulcan cannon}

The vulcan cannon fires \emph{very} high velocity bullets at a high
rate, which reach their target almost instantaneously.  However, each
bullet does not do as much damage as some other weapons.

\noindent
\emph{Ammunition}: vulcan ammo.

\noindent
\emph{Note}: At the moment the Vulcan cannon looks like an AK-47.
This will be fixed in the future.

\subsubsection*{Minigun}

The minigun fires bullets at a high rate.  The bullets do more damage
than vulcan cannon bullets.

\noindent
\emph{Ammunition}: bullets.

\subsubsection*{Bow}

The bow fires arrows, which explode on impact.

\noindent
\emph{Ammunition}: explosive arrows.

\subsubsection*{Rocket launcher}

Rockets do significantly more damage than explosive arrows, however
they take longer to reload.

\noindent
\emph{Ammunition}: rockets.

\subsubsection*{Sniper rifle}

Sniper rifles fire slugs that can penetrate walls, and each slug does
a large amount of damage.  However, reloading the rifle takes a rather
long time.  Since the rifles are designed for sniping, they are also
equipped with a scope that allows you to see further than usual.
Excessive use of the scope also leaves you prone to ambush.

\noindent
\emph{Ammunition}: slugs.

\subsubsection*{Mine}

Unlike other weapons, mines cannot be selected.  They are dropped by
pressing the right mouse button.  After a short time, they will be
automatically armed and then will detonate as soon as a player comes
close to it.  They will also detonate if they are damaged.


\subsection{Other items}

\subsubsection*{Armour}

Armour items come in three flavours: brown, blue and purple.  Picking
up an armour item will increase your armour points, up to a maximum of
50.  Armour reduces the amount of damage done by bullets and
explosions by half.

\subsubsection*{Burger, cola, chocolate bars, pizza, medkit}

These increase your health points by different amounts, up to a
maximum of 100 health points.

\subsubsection*{Bloodlust}

The bloodlust item temporarily doubles the damage your weapons do.
Mines are unaffected.

\subsubsection*{Lightamp}

The lightamp item temporarily gives you night vision goggles that help
you see better.  It will not affect how other players see you.

\subsubsection*{Backpack}

A backpack is dropped when a player dies, containing half of their
ammunition (but no weapons).  Mines are not retained.

\subsubsection*{Red codial}

\emph{Does nothing yet.}



%----------------------------------------------------------------------
\section{Options}

These are options in the menu:

\subsubsection*{Screen resolution}

Select the resolution you wish to play the game in.  Note that the
game is always rendered at 320x200 and only stretched up to larger
resolutions as required, so try not to use 640x400 or 640x480 unless
your video card drivers don't support 320x200.  The menu system always
runs in 640x400 or 640x480.  Only 15- and 16-bit colour depths are
supported.

\subsubsection*{Stretch method}

If you are using a high resolution video mode you have three choices
of how to stretch the original low-res game image.  The choices are
\emph{plain} (fastest), \emph{Super 2xSaI} and \emph{SuperEagle}.

\subsubsection*{SFX}

Choose the sound effects volume.

\subsubsection*{Music}

Choose whether to play music modules in the background, and the music
volume.  Music can take a considerable amount of CPU power to play.
Supported module formats are XM, IT, S3M, and MOD.  You can place
additional modules into the \path{data/music} directory and edit the
\texttt{.txt} files.

\subsubsection*{Gamma}

Adjust the brightness of the game screen.  \emph{You must restart the
game for this setting to take effect.}



%----------------------------------------------------------------------
\section{Editor}

\emph{Undocumented yet.}


%----------------------------------------------------------------------
\section{Known bugs}

\begin{itemize}

\item Sometimes the character becomes stuck on the floor and cannot
  jump.  Workaround: fall down, then jumping ability will be reset.

\item Segfaults when client attempts to run without the map in use.

\item Sometimes suicide is counted twice or more.

\end{itemize}



%----------------------------------------------------------------------
\section{Credits}

Red Pixel 2 was written by Peter Wang, with graphics David Wang.
Kaupo Erme and Oskar Aitaja contributed 3d art and music respectively.
A number of libraries were used, including:

\begin{description}
\item [Allegro] \url{http://alleg.sf.net/}
\item [Libnet] \url{http://libnet.sf.net/}
\item [Lua] \url{http://www.lua.org/}
\item [DUMB] \url{http://dumb.sf.net/}
\end{description}

\noindent
The Red Pixel 2 web site is at \url{http://redpixel.sf.net/}


\end{document}
